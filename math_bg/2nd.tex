\documentclass{article}
\begin{document}
	So, our equation becomes \\
	0 =	$\int_{a}^b (a \frac{dw}{dt} \frac{du}{dx} - wf) dx -[wa \frac{du}{dx}]_{b}^a$\\
	  = 	$\int_{a}^b$ (a $\frac{dw}{dt}$ $\frac{du}{dx}$ - wf$) dx$ - wa$ \frac{du}{dx}\biggr$|_{x=L} + wa$ \frac{du}{dx}\biggr|_{x=0}\\
	  
	  Now n_x=-1$ at x= 0 and n_x=1$ at x= L\\
	  So L,\\
	  0 = 	$\int_{a}^b$ (a$ \frac{dw}{dt}$ $\frac{du}{dx}$ - wf$) dx$ - $n_x wa$ $\frac{du}{dx}$\biggr$|_{x=L} - n_x wa$ $\frac{du}{dx}$\biggr$|_{x=0}\\
	    = 	$\int_{a}^b$ (a$ \frac{dw}{dt}$ $\frac{du}{dx}$ - wf$) dx$ - (wQ)_0 - (w$\theta$)_L\\
	    
	    Finally we impose essential boundary conditions.\\
	    We require that weight function vanishes at boundary where essential boundary conditions are specified.\\
	         i.e;  u(0)= u_0 	-----> w(0)=0\\
	    Thus we get,\\
	        $\int_{0}^{L}(a \frac{dw}{dx} \frac{du}{dx}- wf)dx -w(L)Q_L\\
	        
	   The dependent variable of problem expressed in same form as weight function (w) appears  in the boundary term is called primary variable (PV). Specfying PV is essential boundary condition (EBC).\\
	   
	   In our case W is present in boundary term . So, our PV is u.\\
	   It could be $\frac{du}{dx}$ for higher order differential equation.\\
	   The secondary variable is denoted by Q and \\
	   Q= a $\frac{du}{dx}$n_x\\
	   where $n_x$ is angle between x-axis and normal to boundary.\\
	   
	   For 1D problem, 
	   	n_x= -1$ at left and	n_x=  1$ at right.\\
	   	
	   	We have used the fact that \\
	   	$\int_{a}^{b}$(w$ \frac{du}{dx})dx = -\int_{a}^{b} v dw$ +[wv]_a^{b}$\\
	   	
	   	and considered v = -a$\frac{du}{dx}$\\
	   	
	   	Note that, we take care of boundary conditions which maybe
	   	\begin{itemize}
	   		\item natural
	   		\item essential
	   \end{itemize}	
	  
	  
\end{document}