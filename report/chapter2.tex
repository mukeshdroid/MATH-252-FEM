
\chapter{METHODOLOGY/MODEL EQUATION}

\section{{\bf{Theoretical/Conceptual Framework}}}
{\bf\color{red}In this section, the author(s) should describe the theoretical/mathematical principles behind the whole work relative to the project. The information collected from literature review shall be relevant in this section.
}
%part assigned to samrajya

Consider\\
\begin{eqnarray}
	-\frac{d}{dx} \left[a(x)\frac{du}{dx} \right] = f(x)  \quad\quad  \text{for} \quad\quad  0<x<L.\\
\end{eqnarray}
for $u(x)$ subject to the boundary conditions\\
\begin{eqnarray}
	u(0)= u_0 \\
	\vspace{1cm}
	a(x)\frac{du}{dx}\biggr|_{x=L} = Q_L	
\end{eqnarray}
where $a(x)$ and $f(x)$ are known functions and 
$u_0$ and $Q_L$ are known values.
In case of bar (which is a axially loaded structure)\\

\begin{eqnarray*}
	u = \text{displacement}\\
	a(x)= \text{EA (stiffness)}\\
	f = \text{distributed axial force}\\
	Q_L= \text{axial load}\\
\end{eqnarray*}

Now, we want an approximation of u(x) in the form
\begin{eqnarray}
	u(x)\approx u_N(x) = \sum\limits_{j=1}^N c_j\phi_{j} + (x) +\phi_{0}(x)
\end{eqnarray}

%%page 2 of copy

 Substituting  $U_N$(x) in our Differential Equation,\\
 
 \begin{eqnarray}
 	-\frac{d}{dx} \left[a(x)\frac{dU_N}{dx} \right] = f(x)  \quad\quad  \text{for} \quad\quad  0<x<L.
 \end{eqnarray}

If this equally holds for all $x \in [0,L] $ the solution is exact. Since, we assume it only as an approximation.
we define Residual Function $R(x,c_1, c_2, c_3,...,c_N)$ as
\begin{eqnarray}
	R = -\frac{d}{dx} \left[a(x)\frac{dU_N}{dx} \right] - f(x)
\end{eqnarray} 

We have R $\neq$0 $ \forall x  \in [0, L]$  as a $U_N$ is only a approximation to solution.\\

Now we have various ways to minimize R in some senses over the domain to make this approximation close to the actual solution\\
\begin{enumerate}
	\item Collocation Method\\
	It forces that R is zero at selected N points of the domain.\\
	i.e; 
	\begin{eqnarray}
		R(x,c_1, c_2, c_3,...,c_N) = 0  \quad\quad\quad \forall x=x_i, i = 1,2,...,N
	\end{eqnarray}

	\item Least Square Method
		\begin{eqnarray}
			\frac{\partial}{\partial c_j} \int_{0}^{L} R^2 dx = 0 \quad\quad\quad \forall i = 1,2,...,N
		\end{eqnarray}
	
	\item Weighted Residual Method
		we desire that 
			\begin{eqnarray}
				\int_{0}^{L} w_i(x) R dx = 0 \quad\quad\quad \forall i = 1,2,...,N
			\end{eqnarray}
		where $w_i(x)$ are N linearly independent functions called weight functions.
\end{enumerate}

%page 4
	Now we convert our differential equation and boundary condition to weak form.\\
\begin{itemize}
	\item Step 1 : \\
	We write the weighted integral statement, i.e;\\
	\begin{eqnarray}
		\int_{0}^{L}  w \left[ -\frac{d}{dx} \left( a(x)\frac{du}{dx} \right) - f \right] dx = 0
	\end{eqnarray}

	It is equivalent to differential equations and does not include any boundary conditions and thus the variable $u$ must be differentiable to as many order as is required by the differential equation.
	\item Step 2\\
	Weakening Differentiability conditions
	\begin{eqnarray}
		\int_{0}^{L}  \left( w \left[ -\frac{d}{dx} \left( a(x)\frac{du}{dx} \right) \right] - wf \right)  dx = 0
	\end{eqnarray}
	Integrating by parts, we get,
	\begin{eqnarray}
		\int_{0}^{L} \left[ a \frac{dw}{dx} \frac{du}{dx} - wf \right] dx  - \left[ wa \frac{du}{dx}\right]_0^L = 0
	\end{eqnarray}

%%page 5 of copy

	we have used the fact that 
	\begin{eqnarray}
		\int_{a}^{b} \left( w \frac{dv}{dx} \right)dx  =  -\int_{a}^{b} v dw + \left[wv\right]_0^L = 0
	\end{eqnarray}

by considering,
	\begin{eqnarray}
		v = -a\frac{du}{dx}
	\end{eqnarray}


	
\end{itemize}
%part after assigning to samrajya

\begin{eqnarray}
	B(w,u) = \int_{0}^{L} a \frac{dw}{dx} \frac{du}{dx} dx \\
	l(w) = \int_{0}^{L} wf dx + w(L) Q_L
\end{eqnarray}

Hence, our weak form can be written as

\begin{eqnarray}
	0 = B(w,u) - l(w)
\end{eqnarray}
or ,
\begin{eqnarray}
	B(w,u) = l(w)
\end{eqnarray}

This is the variational form of the problem that is associated with our ODE and its boundary conditions.

When the differential equation is linear and of even order, the resulting weak form will have symmetric bi-linear form in u and w.
 	

% Choosing approximate functions

Let us now choose the approximate solutions that satisfy the two conditions for primary variables as other conditions are already included in the weak form. 
i.e,

\begin{eqnarray}
	u_h^{e} (x_a) = u_1^{e} \\
	u_h^{e} (x_b) = u_2^{e}  
\end{eqnarray}
let,
\begin{eqnarray}
	u_h^{e} (x) = c_1 + c_2 x
\end{eqnarray}

Then by the conditions,

\begin{eqnarray}
	u_h^{e} (x_a) = c_1 + c_2 x_a = u_1^{e}\\
	u_h^{e} (x_b) = c_1 + c_2 x_b = u_2^{e}
\end{eqnarray}

writing in matrix form, we obtain
\begin{eqnarray}
\begin{bmatrix}
	u_1^{e}\\
	u_2^{e}
\end{bmatrix}
=
\begin{bmatrix}
	1 & x_a\\
	1 & x_b
\end{bmatrix}
\begin{bmatrix}
	c_1^{e}\\
	c_2^{e}
\end{bmatrix}
\end{eqnarray}

We can rewrite the equations as 
\begin{eqnarray}
	\begin{bmatrix}
		c_1^{e}\\
		c_2^{e}
	\end{bmatrix}
	= \frac{1}{x_b - x_a}
	\begin{bmatrix}
		x_b & -x_a\\
		-1 & 1
	\end{bmatrix}
	\begin{bmatrix}
		u_1^{e}\\
		u_2^{e}
	\end{bmatrix}
\end{eqnarray}

Hence we get,

