
%\addcontentsline{toc}{chapter}{\bf cHA}

%\begin{center}
%	{\Large{\bf{CHAPTER 1}}}\\
%\end{center}

%\addcontentsline{toc}{chapter}{\bf Acknowledgements}
%\addcontentsline{toc}{chapter}{\bf ACKNOWLEDGEMENTS}

%\begin{center}
%	{\Large{\bf{CHAPTER 1}}}\\
%\end{center}

\chapter{MOTIVATION/INTRODUCTION}

\section{{\bf{Context/Rationale/Background}}}
{\bf\color{red}In this section, the author(s) shall discuss the  historical background related to the work along with the introduction of the topic in brief.}






%%%%%%%%%%%%%%%%%


\section{{\bf{Objectives}}}
{\bf\color{black}
\begin{enumerate}
 \item To set up differential equations with variable boundary conditions. 
 \item  To learn to perform numericals manually and then implement
 those in a high level programming language
 \item To learn about FEM, its variations, and its application to various
 mechanical problems.
 \item Visualization and Project Writing.
\end{enumerate}
}

\section{\bf Significance/Scope}
{\bf\color{red}In this section, the author(s) shall discuss the relevance and the underlying problems that may rise to the need to do this project/work.}

The use of numerical methods is prevalent in all fields of science and technology today. Our project focuses on one powerful numerical tool known as FEM which is theoretically sound and computationally effiecent. The basic ideas of such tools which one is sure to encounter either in higher studies or at a very 

\section{\bf Limitations}
{\bf\color{red}In this section, the author(s) should enlist or elaborate the possible limitations of the project or the difficulty foreseeable during the work.}
The project only focuses on application of FEM to a single differential equation and thus 
