
%\addcontentsline{toc}{chapter}{\bf cHA}

%\begin{center}
%	{\Large{\bf{CHAPTER 1}}}\\
%\end{center}

%\addcontentsline{toc}{chapter}{\bf Acknowledgements}
%\addcontentsline{toc}{chapter}{\bf ACKNOWLEDGEMENTS}

%\begin{center}
%	{\Large{\bf{CHAPTER 1}}}\\
%\end{center}

\chapter{MOTIVATION/INTRODUCTION}

\section{{\bf{Context/Rationale/Background}}}
{\bf\color{red}In this section, the author(s) shall discuss the  historical background related to the work along with the introduction of the topic in brief.}






%%%%%%%%%%%%%%%%%


\section{{\bf{Objectives}}}
{\bf\color{black}
\begin{enumerate}
 \item To set up differential equations with variable boundary conditions. 
 \item  To learn to perform numericals manually and then implement
 those in a high level programming language
 \item To learn about FEM, its variations, and its application to various
 mechanical problems.
 \item Visualization and Project Writing.
\end{enumerate}
}

\section{\bf Significance/Scope}


The use of numerical methods is prevalent in all fields of science and technology today. Our project focuses on one powerful numerical tool known as FEM which is theoretically sound and computationally efficient. The basic ideas of such tools which one is sure to encounter later will be very beneficial. This project also opens up the world of variational calculus and its applications which are generally not covered in undergraduate mathematics.

The computer implementation of a package that implements the algorithm for solving problem using FEM while handling inputs and visualizing the output is a stark contrast from the dummy math problem that are usually used in computer programming classes to teach concepts of general programming rather than mathematical programming. This project imparts the skill to convert mathematical knowledge into efficient and all around packages that solve problems that are based on real world applications and are similar to ones encountered later at work in industries or academia.

\section{\bf Limitations}

The project only focuses on application of FEM to a single differential equation and thus despite being a great starting point for diving into the world of finite element method, it lacks behind in providing full demonstration of its potential. We have applied FEM to solve 2D truss structures. The computer implementation is also limited to this subset of problems and can work with only 2D trusses.
