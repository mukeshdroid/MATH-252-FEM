\addcontentsline{toc}{chapter}{APPENDIX}
\chapter*{APPENDIX}

We will attach all the code created for the module of this packet in this chapter. The updated version can be found at \url{https://github.com/mukeshdroid/trussfem}

\subsection*{node.py}
\begin{lstlisting}[language=Python , basicstyle=\linespread{0.75}\listingsfont]
#class for nodes
import math
import numpy as np

class node:

# init method or constructor
#nan values are used to indicate if the value needs to be determined.
def __init__(self, num, pos_x , pos_y , dis_x , dis_y, f_x , f_y):
self.num = num
self.pos_x = pos_x
self.pos_y = pos_y
self.dis_x = dis_x
self.dis_y = dis_y
self.f_x = f_x
self.f_y = f_y

def print(self):
print('num = ', self.num , '\n' , 'pos_x =', self.pos_x , '\n pos_y = ' , self.pos_y , '\n dis_x = ' , self.dis_x , '\n dis_y = ', self.dis_y , '\n f_x = ', self.f_x, '\n f_y = ', self.f_y)
return 0

def value(self):

return [self.num , self.pos_x , self.pos_y , self.dis_x , self.dis_y, self.f_x, self.f_y]
\end{lstlisting}

\subsection*{ele.py}
\begin{lstlisting}[language=Python , basicstyle=\linespread{0.75}\listingsfont]
#class for elements
import math
import numpy as np
import node

class ele:

# init method or constructor 
#nan values are used to indicate if the value needs to be determined.
def __init__(self, num, length , area , ym , theta, node_a , node_b):
self.num = num
self.length = length
self.area = area
self.ym = ym
self.theta = theta
self.node_a = node_a
self.node_b = node_b

def print(self):
print('num = ', self.num , '\n' , 'length=', self.length , '\n area = ' , self.area , '\n youngs modulus = ' , self.ym , '\n theta = ', self.theta , '\n node_a = ', self.node_a, '\n node_b = ', self.node_b)
return 0

def value(self):
return [self.num, self.length , self.area , self.ym , self.theta , self.node_a.num, self.node_b.num]

def stiff(self):
ang = self.theta
c2 = (math.cos(ang))**2
cs = math.sin(ang) * math.cos(ang)
s2 = (math.sin(ang))**2
mat = np.array([[c2 , cs , -c2 , -cs],
[cs , s2 , -cs , -s2],
[-c2 , -cs , c2 , cs],
[-cs , -s2 , cs , s2]])
return ((self.ym * self.area) / self.length) * mat
\end{lstlisting}

\subsection*{node.py}
\begin{lstlisting}[language=Python , basicstyle=\linespread{0.75}\listingsfont]
#class for nodes
import math
import numpy as np

class node:

# init method or constructor
#nan values are used to indicate if the value needs to be determined.
def __init__(self, num, pos_x , pos_y , dis_x , dis_y, f_x , f_y):
self.num = num
self.pos_x = pos_x
self.pos_y = pos_y
self.dis_x = dis_x
self.dis_y = dis_y
self.f_x = f_x
self.f_y = f_y

def print(self):
print('num = ', self.num , '\n' , 'pos_x =', self.pos_x , '\n pos_y = ' , self.pos_y , '\n dis_x = ' , self.dis_x , '\n dis_y = ', self.dis_y , '\n f_x = ', self.f_x, '\n f_y = ', self.f_y)
return 0

def value(self):

return [self.num , self.pos_x , self.pos_y , self.dis_x , self.dis_y, self.f_x, self.f_y]
\end{lstlisting}

\subsection*{trusssolve.py}
\begin{lstlisting}[language=Python , basicstyle=\linespread{0.75}\listingsfont]
import sys
from node import node
from ele import ele
import turtle
import numpy as np
import math
import copy
import pyautogui
import time
import csv
import os

node_file = sys.argv[1] 
ele_file = sys.argv[2]

class truss:
node_list = []
ele_list = []
dis_list = []
f_list = []
GK = []

def __init__(self):

with open(node_file) as csvfile1:
reader = csv.reader(csvfile1)
for row in reader:
num = int(row[0])
pos_x = float(row[1])
pos_y = float(row[2])
dis_x = np.nan if row[3] == 'nan' else float(row[3])
dis_y = np.nan if row[4] == 'nan' else float(row[4])
f_x = float(row[5])
f_y = float(row[6])
temp1 = node(num, pos_x, pos_y, dis_x, dis_y, f_x, f_y)
self.node_list.append(temp1)

with open(ele_file) as csvfile2:
reader = csv.reader(csvfile2)
for row in reader:
num = int(row[0])
length = float(row[1])
area = float(row[2])
ym = float(row[3])
theta = math.pi * float(row[4][3:])
node_a = self.node_list[int(row[5]) - 1]
node_b = self.node_list[int(row[6]) - 1]
temp2 = ele(num, length, area, ym, theta, node_a, node_b)
self.ele_list.append(temp2)

def globalstiff(self):
# dimension of global matrix
dim = 2 * len(self.node_list)
eles = self.ele_list
# generate and store the element-wise stiffness matrices

GK = np.zeros((dim, dim))

for e in eles:
i = 2 * e.node_a.num - 2
j = 2 * e.node_a.num - 1
k = 2 * e.node_b.num - 2
l = 2 * e.node_b.num - 1
e_stiff = e.stiff()

index = [i, j, k, l]
index2d = [(a, b) for a in index for b in index]
d = {i: 0, j: 1, k: 2, l: 3}
for p, q in index2d:
GK[p][q] = GK[p][q] + e_stiff[d[p]][d[q]]
self.GK = GK

def solve(self):
# generate global matrix by calling globalstiff method
self.globalstiff()

dis_list = []
for n in self.node_list:
dis_list.append(n.dis_x)
dis_list.append(n.dis_y)

dis = np.array(dis_list)

f_list = []
for n in self.node_list:
f_list.append(n.f_x)
f_list.append(n.f_y)

f = np.array(f_list)

#check for undetermined values in dis and create linear eqns

GK_dis =  copy.deepcopy(self.GK)
f_dis =  copy.deepcopy(f)

#get a list of rows to remove
del_row = []

index_list = list(range(0,2*len(self.node_list)))
for i in range(0,2*len(self.node_list)):
if not np.isnan(dis_list[i]):
del_row.append(i)
index_list.remove(i)

#remove the rows that have displacement given
GK_dis = np.delete(GK_dis,del_row,0)
f_dis = np.delete(f_dis,del_row,0)

#before deleting the columns we subratct these from force vector
for i in del_row:
f_dis = f_dis - dis_list[i] * GK_dis[:,i]

#delete the columns that are due to the displacements that are determined
GK_dis = np.delete(GK_dis,del_row,1)   

ans_dis = np.linalg.solve(GK_dis, f_dis)

for i in range(0, len(ans_dis)):
dis[index_list[i]] = ans_dis[i]
self.dis_list = dis

ans_force = np.dot(self.GK, dis)
self.force_list = ans_force

k = 0
for i in self.node_list:
i.dis_x = self.dis_list[k]
k = k + 1
i.dis_y = self.dis_list[k]
k = k + 1

#print(self.dis_list)
#print(self.force_list)

def visualize(self, height=560, width=1300, grid=12, speed=7, delay=2):
try:
width, height = pyautogui.size()
except:
None

turtle.title("Visualization")
turtle.setup(width, height)
turtle.bgcolor("black")
ratio = height / width
turtle.setworldcoordinates(-2, -2 * ratio, grid, ratio * grid)

t = turtle.Turtle()
t.speed(speed)
t.hideturtle()
t.pensize(5)
t.pencolor("blue")

for ele in self.ele_list:
t.goto(ele.node_a.pos_x, ele.node_a.pos_y)
t.pendown()
t.goto(ele.node_b.pos_x, ele.node_b.pos_y)
t.penup()

t.speed(0)
for node in self.node_list:
t.penup()
t.goto(node.pos_x, node.pos_y)
t.pendown()
t.dot(15, "red")

t.penup()
t.goto((ele.node_a.pos_x + ele.node_a.dis_x), (ele.node_a.pos_y + ele.node_a.dis_y))
t.pendown()

# def fun():
#    return None 
# turtle.onclick(fun, btn=1, add=None)
time.sleep(delay)

t.speed(speed)
# after solving
t.pencolor("green")
for ele in self.ele_list:
t.goto((ele.node_a.pos_x + ele.node_a.dis_x), (ele.node_a.pos_y + ele.node_a.dis_y))
t.pendown()
t.goto((ele.node_b.pos_x + ele.node_b.dis_x), (ele.node_b.pos_y + ele.node_b.dis_y))
t.penup()

t.speed(0)
for node in self.node_list:
t.penup()
t.goto((node.pos_x + node.dis_x), (node.pos_y + node.dis_y))
t.pendown()
t.dot(15, "yellow")
turtle.Screen().exitonclick()

def writeoutput(self):
with open('./csv_files/output.csv' , 'w') as csvfile3:
writer = csv.writer(csvfile3)
writer.writerow(['num','pos_x','pos_y', 'dis_x','dis_y',  'f_x', 'f_y'])
for node in self.node_list:
writer.writerow(node.value())
writer.writerow(['num', 'length' , 'area' , 'ym', 'theta', 'node_a' , 'node_b'])
for ele in self.ele_list:
writer.writerow(ele.value())

truss1 = truss()
truss1.solve()
truss1.writeoutput()
truss1.visualize()
\end{lstlisting}